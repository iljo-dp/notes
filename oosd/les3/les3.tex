Pseudocode
Het cshrijven in normale taal van een algoritme. Het is een manier om een algoritme te beschrijven zonder dat je je druk hoeft te maken over de syntax van een bepaalde programmeertaal.

Niet bezigzijn met de specifieke syntax van een bepaalde programmeertaal, maar met de logica van het algoritme.

Als punten groter dan of gelijk aan 60;
	print geslaagd
break;

Controlestructuren;
Sequentie, selectie en iteratie.
Sequentie; een voor een, in sequentie. Opeenvolgend. Van boven naar onder.
Selectie; keuze maken tussen twee of meer alternatieven. Als dit, dan dat.

Methoden;
Het gebruiken van een nieuwe private void naam()
om zo een blok code te kunnen hergebruiken. en zo te abstracteren en gebruiken.

if -> 1 beslissing
if else -> 2 beslissingen
if else if else -> 3 beslissingen
conditie, dan, anders, dan, anders, dan
