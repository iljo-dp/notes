\documentclass{article}
\author{iljo De Poorter}
\title{Les 2 it-fundamentels}
\begin{document}
\maketitle
\section{Conversie tussen de stelsels}
van binair naar octaal is binair opsplitsen in groepjes van 3 en dan omzetten naar octaal. Dus 100110 = 100 110 = 46
1001100,10011 =  0010 1100 100 100 = 2344

\subsection{Binaire conversie}
1bit = 2 mogelijkheden, 0 en 1
2 bits = 4 mogelijkheden, 00,01,10,11

n bits = 2 tot de n mogelijkheden

Andere manier van binair omrekenen. 10110101 = 1*7$^2$ + 0*6$^1$ + 1*5$^0$ + 1*$^4$ + 0= 181

Commagetal naar normale cijfers.
\begin{enumerate}
\item 1			0.5
\item 0.5		0.25
\item 0.25		0.125
\item 0.125		0.0625
\item enz...
\end{enumerate}
0.875 = 0,875 - 0,5 = 0,375 - 0,25 = 0,125 - 0,125(4) = 0
Grootste macht zoeken die kleiner is dan het getal. 2$^{-1}$ = 0.5

(0,3)10 = (...)2
0,3*2 = 0,6
0,6*2 = 1,2
0,2*2 = 0,4
0,4*2 = 0,8
0,8*2 = 1,6
0,6*2 = 1,2
= 0,010011(altijd het cijfer VOOR de komma)

\subsection{gythiel}
Je neemt je getal, Bv 131, en deelt het door 2
als er rest is is het getal 1, geen rest is het 0
Dus
131/2 = 65 rest 1
65/2 = 32 rest 1
32/2 = 16 rest 0
16/2 = 8 rest 0
8/2 = 4 rest 0
4/2 = 2 rest 0
2/2 = 1 rest 0
1/2 = 0 rest 1
Dus 131 = 10000011
\end{document}

