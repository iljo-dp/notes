Poorten, Werkt met logica. De poorten zijn de basis van de logica
Bij poorten is er een input, een 0 of een 1. Men spreekt dus van waar en onwaar, 1 en 0.

A B | output
0 0 | 0
0 1 | 0
1 0 | 0
1 1 | 1
Zo kunnen we werken met een waarheidstabel. We werken dus met een AND poort. 
2 inputs is 4 mogelijkheden. 3 inputs is 8 mogelijkheden. 4 inputs is 16 mogelijkheden.
Een en poort wordt getekdn als een hoofdletter =D-, een Niet poort wordt getekend als een -|>°-. En een of pport als een =)>-.

Basispoorten, vervolg.
De logische NIET/NOT komt overeen met complementeren in de logica. 0 komt binnen, output 1, en omgekeerd. 

De EN poort, AND, is een poort met 2 inputs. De output is 1 als beide inputs 1 zijn. Anders is de output 0.

De OF/OR poort, als er minstens 1 eentje is, is de output 1. Anders is de output 0.

XOR poorten ))>. Exclusieve OF. Als er 1(max) eentje is, is de output 1. Anders is de output 0. Dus 1 1 = 0, maar 1 0 = 1

NAND poorten. Niet EN. Als er minstens 1 nul is, is de output 1. Anders is de output 0.

NOF, 00 = 1, anders 0.

TRI-state bufferr

Dit 
