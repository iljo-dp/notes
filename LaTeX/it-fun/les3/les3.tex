\documentclass{article}
\author{iljo De Poorter}
\title{Les 2}
\begin{document}
\maketitle
\section{Bewerkingen in het binair stelsel}
Optellen in het decimaal
Optellen werkt hetzelfde als bij het decimaal, maar als je dus 11 + 10 doet krijg je dus 101
\subsubsection{Opetellen in het decimaal stelsel/zelfde methode als binair}
Werkwijze, scrijf de twee getallen onder elkaar met de komma's onder elkaar.
Tel cijfer per cijfer bij elkaar op, van rechts naar links.
Draag indien nodig een 1 over naar de volgende kolom.

\subsection{oefeningen}
\begin{enumerate}
\item (1011)2 + (10101)2 = (100000)2
\item (110,11)2 + (10,101)2 = (1001,011)2
\item (23,25)10 + (40,5)10 = (...)2 = (101111,01)2
\end{enumerate}
Zie blad
\section{complementen}
(1011) = 11
(0100) = 1(1011)
(00001011)2 = 11
(11110100)2 = 7(11)
Complemtenten draaien alle 0'en en 1'en om. 
127+87 = 214
(01111111)2 + (01010111)2 = (11001110)2
127-87 = 40
(01111111)2 - (01010111)2 = (00101000)2
106 = (01101010)2
106 + 172 = 278
(01101010)2 + (10101100)2 = (100110110)2
Teken is gewoon het getal in binaire vorm.
Exces-127 is het binair van het getal plus 01111111
2's complement is het het binair met een positief getal en het inverten van elk getal + 1 voor een negatief getal
Dus Vb: 106 = (01101010)2, 01111111+01101010 = 11100101, 01101010
\section{Oefeningen}
	Teken + abs		excess-127		2's complement
	00001010		10001001		00001010
	10001101		01110010		11110011
	00101011		
106	01101010		11101001	        01101010
-106	10010110		00010101		10010110
127	01111111		11111110		0111111
-127	11111111		00000000		10000001
128	overflow		11111111		overflow
-128	overflow		overflow		10000000


\end{document}

