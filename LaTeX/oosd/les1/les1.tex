\documentclass{article}
\author{iljo De Poorter}
\title{Object oriented software Development}
\begin{document}
\maketitle
\section{Intro}
Doelstellling:

\subsection{inleiding}
een van de meest gebruikte programeertalen ooit.

Draait op 51 miljard devices.

Java ondersteund oop, generic programming, functional programming.

We gebruiken java 17 SE, een stabiele versie van java. Heeft alles wat we nodig hebben en is dus tevens stabiel.

ontstaan;
1991, Sun microsystems
1995, java 1.0. 


Meeste software toen werdt gebruikt per computer. Java was een van de eerste talen die het mogelijk maakte om software te schrijven die op verschillende devices kon draaien.

\subsection{Java Virtual Machine}
Elk java-programma wordt door een compiler vertaalt naar bytecode. Deze bytecode wordt dan door de JVM vertaalt naar machinetaal. De JVM is dus een soort van vertaler.

De JVM voert alle code uit, en vertaalt het naar machine code. De JVM is dus een soort van vertaler.

Cui, console user interface. Een programma dat enkel werkt met tekst.

Variabelen in JAVA. Vanaf dat we een variabele aanmaken wordt deze toegekend aan de stack. Op elk 'deeltje' van de stack wordt er dan een soort variabele toegekent..
Elke variebele, heeft een naam, een type en een waarde. De naam is de naam van de variabele, de waarde is de waarde van de variabele, en het type is het type van de variabele.

\end{document}
