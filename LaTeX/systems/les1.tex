\documentclass{article}
\author{iljo De Poorter}
\title{My first {\LaTeX} document}
\begin{document}
\maketitle
\section{Embeded vs echte systems}
Embeded systems zijn zeer kleine chipsets met alles wat het nodig heeft erop en eraan. Een headset is bv een imbeded system. Het heeft een chipje die maar 1 ding doet.

Terwijl andere chip's zoals gsm/computers meerdere complexere dingen kunnen verwerken.

\subsection{Hedendaagse tech}
De dominante tech van vandaag is ditaal. Het is een input output machine. Je geeft iets in het het geeft output
\subsection{Digitale tech}
Alles via electronica, zonder elecktriciteit kan je niets doen.
Nadeel; De hitte afkomst van electriciteit is zeer slecht, door die hitteafkomst kunnen de batterijen ontplofen en op zijn minst performantie verliezen(de chip).

Digitaal; Maakt gebruik van discrete, discontinue waarde.
Binair; maakt gebruik van slechts twee waarden.

\subsection{Analoog vs digitaal}

Analoog; elke spanning-of stroomwaarde tussen twee grenzen heeft een betekenis.

Digitaal; Slechts twee waarden, 0 of 1.

\subsection{Waarom digitaal?}
Stabiliteit, kwaliteit, reproduceerbaarheid, flexibiliteit, goedkoop.
Stabiel in de tijf en bij bewerkingen of versturen.
Zelf kiezen,
Goedkoop, Digitale schakelingen zijn eenvoudigere en lenen zich gemakelijker tot hoge integratie op integrated circuits.
Door het digitale kunnen we meerdere dingen doen op 1 toestel.
Dezelfde hardware kan dus worden gebruikt voor verschillende dingen.
Alles op 1 apparaat.

\subsubsection{waarom digitaal}
Redundantie, encryptie en compressie zijn eenvoudig toe te passen.

betrouwbaarheid.

Klein, draagbaar dankzij de miniaturisatie van de componenten.

Alles wordt gedigitaliseerd.

\subsection{kwantum}
qubit, kan 1 of 0 zijn. Expirimenteel.
Voor zeer grote berekeningen te doen.

\section{hoofdstuk 2}
Wat is een bestuuringssyteem? Hetgene dat de hardware en de software laat samenwerken.
System software, de software die echt nodig is om je OS op te starten. De BIOS, je firmeware. Dan wordt alles in je RAMFS geladen.

Wat is het verscil tussen hardware, kernel en de shell.
Hardware, het fysieke in de computer zelf.
Kernel, de software die de hardware aanstuurt.
Shell, de software die de gebruiker toelaat om met de kernel te communiceren.

\subsection{Keuze van de OS}

Rol, direct toegankelijk door één gebruiker (desktop) of meerdere gebruikers (server).
Gebruik, algemeen (desktop) of specifiek (server). 

Stabiliteit, zijn os-releases bèta(niet getest) of stabiel (getest).

compatibiliteit, is het achterwaards compatibel?

Kost, micorsoft werkt met een jaarlijske licenties.
Apple heeft geen licensie kost voor de software, maar je moet veel te dure hardware kopen.
Linux is veelal gratis terwijl de kost voor support is.

\end{document}



