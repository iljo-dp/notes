\documentclass{article}

\usepackage{titlesec}
\usepackage{titling}
\usepackage[margin=1in]{geometry}


\titleformat{\section}
{\huge\bfseries\lowercase}{}{0em}{}[\titlerule]

\titleformat{\subsection}
{\bfseries\Large}{$\bullet$}{0em}{}

\titleformat{\subsubsection}[runin]
{\bfseries\large}{}{0em}{}[---]


\renewcommand{\maketitle}{
\begin{center}
{\huge\bfseries
\theauthor}

\vspace{.25em}
i@iljo.dev --- https://iljo.dev --- https://github.com/iljo-dp
Iljo De Poorter
\end{center}
}

\begin{document}
\title{Databases Les4}
\author{Iljo De Poorter}

\maketitle

\subsection{Herhaling}
EERD, enhanced entity relationship diagram
 {Optional, AND} (Mandatory, AND) (Optional, OR) (Mandatory, OR)
 Als een PARTENT zwak is, zijn de kinderen ook zwak.
 \section{hoofdstul 5. Relationeel MODEL }
 Veel op veel bestaat niet, en zullen we ook niet meer gebruiken.

 Het relationeel model is een stappenplan dat we constant uit zullen voeren. 

 \subsection{Defenities}
 \subsubsection{Tupel}
 Een georderde lijst van kenmerken, die een entiteit voorstelt. Een rij/record. Bv een beschrijving van een auto., rijen en kolommen. Georderde lijst met attribuutwaarden.
 \subsubsection{Attribuut}
 Een benoemd kenmerk, dat een eigenschap van een entiteit beschrijft. Een kolom. Bv een kleur van een auto. Inhoud van 1 cel
 \subsubsection{Domein}
 De toegestane waarden van een attribuut. Bv een kleur kan rood, blauw, groen, ... zijn.
 \subsubsection{Datatype}
 Elk atribuut is van een bepaald type, dit is afgeleid uit het domein. Bv een kleur is een string.
 \subsubsection{Tupel verzameling}
 Een verzameling van tupels, die dezelfde attributen hebben. Een tabel. Bv een verzameling van auto's. AKA een tabel.

 Veel op Veel valt weg uit een ERD, maar is niet mogelijk in een relationele database.

 Elke tupel is uniek
 Elk attribuut is eenwaardig(maar 1 waarde)
 Elke attribuut is atomair(kan niet worden gespitst)
 Verbanden tussen relaties worden gelegd aan de hand van vreemde sleutels.

 Een 1 op veel locatie staat de vreemde sleutel langs de veel kant.

 \section{mapping}
 \begin{itemize}
\item Mappen is het omzetten van ERD naar relationeel.
\item Elk entiteitstype wordt een tupelverzameling
\item Enkelvoudige attribuuttypes overenmeen
\item samengestelde attribuuttypes opslitsen. Naam --> voornaam famillienaam
\item Maarwaardige attributen opsliptsen in meerdere attribuut types
\item Primaire sleutels bepalden(kijk uit voor zwak)
\item Voor elke relatie tussen entitettypen de vreemde sleutels bepalen.
\item integriteitsregels bepalen voor elke sleuten.
\end{itemize}

\subsection{regels voor het bepalen van de vreemde sleutels}
binair
\begin{itemize}
	\item 1 op veel relatie: vreemde sleutel aan de kant van de veel
    	\item 1 op 1 relatie: als er een optionele kant is, zet hem daar
	\item veel op veel relatie: Werk met een tussentabel(tertaire tabel)
\end{itemize}
Unair
\begin{itemize}

	\item 1 op veel
	\item 1 op 1
	\item veel op veel
\end{itemize}
\subsection{voorbeeld}
Map een 1 op N binaire relatie
stappen;

Elke entiteit wordt een tabel. 
Schrijf dan alle attribuuttypen op.
Vreemde sleutel van een binaire relatie.

Als je een unaire recursieve 1 op N relatie hebt zelf je opnieuw bij de relatie heeftAlsBaas, Baas als een entiteitstype toevoegen.

1 op 1 (binair)
Je wilt de vreemde sleutel zoveel mogelijk invullen. 
Elke firmawagen heeft een personeelslid, maar niet elk personeelslid heeft een firmawagen.
Dus we zetten bij firmawagen de vreemde sleutel van personeelslid, werkenemer, Want dit is altijd ingevuld.
\end{document}
