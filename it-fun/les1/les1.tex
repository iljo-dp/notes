\documentclass{article}
\author{iljo De Poorter}
\usepackage[dutch]{babel}
\title{Eerste les It-FUN}
\begin{document}
\maketitle

\section{Inleiding}
\subsubsection{Voorstellen van getallen}
Gettalen kunnen op verschillende manieren worden voorgesteld.
V, IIII/, 5, 101, 0x5, 0b. is allemaal 5.
\subsubsection{Definities}
Een cijfer is een symbool dat gebruikt wordt bij de voorstelling van getallen.


Decimaal, 0,1,2 tot 9.
Door deze achter elkaar te plaatsen krijgen we getallen.
\subsection{Positionele systemen}
We moeten geen def's van buiten kennen. Maar wel kennen en snappen.
Het is een talstelser waarbij een getal wordt voorgesteld door een reeks symbolen of cijfers. En hun positie is van belang.
3021 = 3² + 10² + 10$^2$ + 10$^0$
\subsection{Binaire stelsel}
Grondtal = 2, verzameling symbolen {0,1}.
Bv. 1011010 = 1*2$^6$ + 0*2$^5$ + 1*2$^4$ + 1*2$^3$ + 0*2$^2$ + 1*2$^1$ + 0*2$^0$ = 90
\subsubsection{IP adressen}
ipV4 is 32 bits. Dat wordt dus veel te lang met en onoverzichtelijk met veel kans op fouten. Daarom doen we dit in een ander talstelsel. Dotted decimal.
\subsubsection{Afspraken en notities}
Als we met een bepaald talstelser werken zetten we er haakjes rond met dan de "base" van het talstelsel. Binair is base 2 dus (10)2. Niet nodig met normaal base 10 systeem wat normaal is voor ons.
\subsubsection{Binair tellen}
Methode.
1. Beginnen bij 0
2. Vervang het laatste cijfer met zijn opvolger
3. Wanneer een 1 verhoogd wordt gaan we een rang hoger met een.
4. Ga naar stap 2.
Dus 
0
1
10
11
100
\subsubsection{Definities}
bit = binary digit
afgekort b
1 cijfer (0 of 1) in het binaire stelsel

Byte = 8 bits
afgekort B

msb = most significant bit
lsb = least significant bit
msb = het eerste cijfer van links
lsb = het laatste cijfer van rechts
\subsection{Octale getallen}
Grondtal = 8
Verzameling symbolen = {0,1,2,3,4,5,6,7}
Voorbeeld is bv (1024)8 = 1*8$^3$ + 0*8$^2$ + 2*8$^1$ + 4*8$^0$ = 532
Ipv ()98 kan er ook gewoon een 0 voor jouw getal staan

\subsubsection{Octaal stelsel}
0,1,2,3,4,5,6,7
Dus 8 is dan 10 en 9 is 11 enz.
0 = 000
1 = 001
2 = 010
3 = 011
4 = 100
enz...
elk cijfer in het octale stelsel kan je voorstellen met 3 bits.
\subsubsection{toepassing}
Bestandpermissies in Linux;
Bestaat uit 3 octale cijfers;
Algemeen; -rwx(owner) rwx(group) rwx(other) bestandsnaam.
111 = 7 = rwx, 101 = 5 = r-x
\subsection{Hexadecimale getallen}
Grondtal = 16
Verzameling symbolen = {0,1,2,3,4,5,6,7,8,9,A=10,B=11,C=12,D=13,E=14,F=15}
Voorbeeld is bv (1024)16 = 1*16$^2$ + 0*16$^1$ + 2*16$^0$ = 4096
Voorbeeld, (A1) = 10*16$^1$ + 1*16$^0$ = 161
\subsubsection{Hexadecimaal stelsel}
0,1,2,3,4,5,6,7,8,9,A,B,C,D,E,F
Dus 10 is A en 11 is B enz.
0 = 0000
1 = 0001
8 = 1000
9 = 1001
F = 1111
10 = 10000
\subsubsection{toepassing}
ipV6 adressen
128 bits, 32 hexadecimale cijfers in groepen van 4

MAC adressen
48 bits, 12 hexadecimale cijfers in groepen van 2
\subsection{Conversie tussen talstelsels}
\subsubsection{conversie met basis van een macht 2}
In geval dat d ebasis van een talstelsel een macht van 2 is, zal een van de symbolen worden voorgesteld door een vast aantal bits.
VB.
Octaal talstelser basis = 8 = 2²

Binair naar hexadecimaal. (11010,0100001)2 = (0001,1010,0100,0010)2 = (1,A,4,2)16 
(7,8A)16 = 0111,100011010 
(4,17)16 = 000100,00110110)2 = 4,056

\subsubsection{oefeningen op conversie}

1. (100110010011)2 = (993)16 = (4623)8

100 110 001 011

4 6 2 3

2. (1001110011110010)2 = 9CF2 = (116362)8

001 001 110 011 110 010

1 1 6 3 6 2

3. (101010111100)2 = ABC = (5274)8

101 010 111 100

5 2 7 4

4. (1FD)16 = (0001 1111 1101)2 = (111111101)2 (voorste nullen schrappen)

5. (CCC)16 = (1100 1100 1100)2

6. (5307)8 = (101 011 000 111)2

7. (7264)8 = (111 010 110 100)2

8. (2A5C)16 = (0010 1010 0101 1100)2 = (000 010 101 001 011 100)2 = (25134)8

9. (243)8 = (0010 100 0011)2 = (010 100 011)2 = (1010 0011)2 = (A3)16

\end{document}
