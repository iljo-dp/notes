\documentclass{article}
\author{iljo De Poorter}
\title{Software analyse les4}
\begin{document}
\maketitle
\section{Domeinmodelen software analyse}
Use case diagram, het bijhouden van alles over de use cases.

Waarom?
\begin{itemize}
\item Om goed te weten wat er aan de hand is met je project.
\item Om te weten wat je moet doen.
\item Visuele representatie van concepten uit de werkelijkheid, die we modeleren en waarvan we de onderlinge relatie tonen.
\end{itemize}
Er bestaan verschillende domeinklassen. Deze gebruiken we tot aan de kern van het probleem te geraken. 

Met hulp van het domeinmodel kan je nog steeds afstemmen met de klant of wij alles goed begrijpen.


Conceptuele klassen; domeinklassen, concepten uit de werkelijkheid die we modelleren.
Doel?
\begin{itemize}
\item Klassen om objecten uit de werkelijkheid te modeleren.
\end{itemize}


Associetienaam, naam van de relatie tussen de klassen.
\end{document}
